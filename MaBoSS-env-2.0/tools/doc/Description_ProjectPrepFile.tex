\documentclass{article}
\usepackage{amsmath, amsthm, amssymb,algorithm2e}
\newcommand{\prob}{\mathbf{P}}


\begin{document}
\section*{MBSS\_ProjectPrepFile.sh}
\subsection*{Aim}
MBSS\_ProjectPrepFile.sh prepares a set of MaBoSS run, according to different modifications of a generic model (mutations, parameter sensitivity, etc.)

\subsection*{How to run the script}
MBSS\_ProjectPrepFile.sh is a shell script that uses a perl library, MBSS\_ProjectPrepFile.pm and two python3 scripts: MBSS\_ProjectPrepSimTrajFig.py and MBSS\_ProjectPrepFilePieChart.py.
Launching MBSS\_ProjectPrepFile.sh is done with the following command:
\begin{verbatim}
MBSS_ProjectPrepFile.sh File.pmbss
\end{verbatim}

MBSS\_ProjectPrepFile.sh must be accessible by command line, as MaBoSS. MBSS\_ProjectPrepFile.pm, MBSS\_ProjectPrepSimTrajFig.py and MBSS\_ProjectPrepFilePieChart.py
must be accessible within an environment variable.

The file ``File.pmbss'' contains the following fields

\begin{itemize}

  \item \begin{verbatim}MABOSS = Executable_name;\end{verbatim}
  \item  \begin{verbatim}BND= bnd_file.bnd;\end{verbatim}
  \item  \begin{verbatim}CFG= cfg_file.cfg;\end{verbatim}

The file File.pmbss can contain the optional fields:

\item    \begin{verbatim}INIT_COND=[Previous_model.bnd,Previous_model_probtraj.csv,#line_number];\end{verbatim}
From a previous run of MaBoSS that uses the file ``Previous\_model.bnd'' and produces a trajectory file ``Previous\_model\_probtraj.csv'', PrepMultiSim uses the line ``\#line\_number'' of this trajectory file as an initial condition in the .cfg file(s) in the project folder.
\item    \begin{verbatim}MUT= First_node Second_Node ...;\end{verbatim}
\item    \begin{verbatim}COMB_MUT=#number_of_possible_combined_mutation;\end{verbatim}
\item    \begin{verbatim}VAR_SENS=[$External_variable1,Suffix_to_add] [$External_variable2,Suffix_to_add]  ...;\end{verbatim}
\item    \begin{verbatim}COMB_VAR_SENS=#number_of_possible_combined_variable_sensitivity;\end{verbatim}
\item    \begin{verbatim}TRAJ_TABLE=yes;\end{verbatim}
\item    \begin{verbatim}STAT_TABLE=[yes,#probability_threshold];\end{verbatim}

\end{itemize} 

\subsection*{Outputs}

According to the file ``File.pmbss'', a project folder ``File'' is created, which contains all necessary inputs and a shell script ``File.sh''. Launching this shell script runs (multiple) MaBoSS and creates output tables and figures (if specified).

\end{document}
